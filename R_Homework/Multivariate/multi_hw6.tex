% Options for packages loaded elsewhere
\PassOptionsToPackage{unicode}{hyperref}
\PassOptionsToPackage{hyphens}{url}
%
\documentclass[
]{article}
\usepackage{amsmath,amssymb}
\usepackage{iftex}
\ifPDFTeX
  \usepackage[T1]{fontenc}
  \usepackage[utf8]{inputenc}
  \usepackage{textcomp} % provide euro and other symbols
\else % if luatex or xetex
  \usepackage{unicode-math} % this also loads fontspec
  \defaultfontfeatures{Scale=MatchLowercase}
  \defaultfontfeatures[\rmfamily]{Ligatures=TeX,Scale=1}
\fi
\usepackage{lmodern}
\ifPDFTeX\else
  % xetex/luatex font selection
    \setmainfont[]{NanumGothic}
    \setmonofont[]{UnShinmun}
\fi
% Use upquote if available, for straight quotes in verbatim environments
\IfFileExists{upquote.sty}{\usepackage{upquote}}{}
\IfFileExists{microtype.sty}{% use microtype if available
  \usepackage[]{microtype}
  \UseMicrotypeSet[protrusion]{basicmath} % disable protrusion for tt fonts
}{}
\makeatletter
\@ifundefined{KOMAClassName}{% if non-KOMA class
  \IfFileExists{parskip.sty}{%
    \usepackage{parskip}
  }{% else
    \setlength{\parindent}{0pt}
    \setlength{\parskip}{6pt plus 2pt minus 1pt}}
}{% if KOMA class
  \KOMAoptions{parskip=half}}
\makeatother
\usepackage{xcolor}
\usepackage[margin=1in]{geometry}
\usepackage{color}
\usepackage{fancyvrb}
\newcommand{\VerbBar}{|}
\newcommand{\VERB}{\Verb[commandchars=\\\{\}]}
\DefineVerbatimEnvironment{Highlighting}{Verbatim}{commandchars=\\\{\}}
% Add ',fontsize=\small' for more characters per line
\usepackage{framed}
\definecolor{shadecolor}{RGB}{248,248,248}
\newenvironment{Shaded}{\begin{snugshade}}{\end{snugshade}}
\newcommand{\AlertTok}[1]{\textcolor[rgb]{0.94,0.16,0.16}{#1}}
\newcommand{\AnnotationTok}[1]{\textcolor[rgb]{0.56,0.35,0.01}{\textbf{\textit{#1}}}}
\newcommand{\AttributeTok}[1]{\textcolor[rgb]{0.13,0.29,0.53}{#1}}
\newcommand{\BaseNTok}[1]{\textcolor[rgb]{0.00,0.00,0.81}{#1}}
\newcommand{\BuiltInTok}[1]{#1}
\newcommand{\CharTok}[1]{\textcolor[rgb]{0.31,0.60,0.02}{#1}}
\newcommand{\CommentTok}[1]{\textcolor[rgb]{0.56,0.35,0.01}{\textit{#1}}}
\newcommand{\CommentVarTok}[1]{\textcolor[rgb]{0.56,0.35,0.01}{\textbf{\textit{#1}}}}
\newcommand{\ConstantTok}[1]{\textcolor[rgb]{0.56,0.35,0.01}{#1}}
\newcommand{\ControlFlowTok}[1]{\textcolor[rgb]{0.13,0.29,0.53}{\textbf{#1}}}
\newcommand{\DataTypeTok}[1]{\textcolor[rgb]{0.13,0.29,0.53}{#1}}
\newcommand{\DecValTok}[1]{\textcolor[rgb]{0.00,0.00,0.81}{#1}}
\newcommand{\DocumentationTok}[1]{\textcolor[rgb]{0.56,0.35,0.01}{\textbf{\textit{#1}}}}
\newcommand{\ErrorTok}[1]{\textcolor[rgb]{0.64,0.00,0.00}{\textbf{#1}}}
\newcommand{\ExtensionTok}[1]{#1}
\newcommand{\FloatTok}[1]{\textcolor[rgb]{0.00,0.00,0.81}{#1}}
\newcommand{\FunctionTok}[1]{\textcolor[rgb]{0.13,0.29,0.53}{\textbf{#1}}}
\newcommand{\ImportTok}[1]{#1}
\newcommand{\InformationTok}[1]{\textcolor[rgb]{0.56,0.35,0.01}{\textbf{\textit{#1}}}}
\newcommand{\KeywordTok}[1]{\textcolor[rgb]{0.13,0.29,0.53}{\textbf{#1}}}
\newcommand{\NormalTok}[1]{#1}
\newcommand{\OperatorTok}[1]{\textcolor[rgb]{0.81,0.36,0.00}{\textbf{#1}}}
\newcommand{\OtherTok}[1]{\textcolor[rgb]{0.56,0.35,0.01}{#1}}
\newcommand{\PreprocessorTok}[1]{\textcolor[rgb]{0.56,0.35,0.01}{\textit{#1}}}
\newcommand{\RegionMarkerTok}[1]{#1}
\newcommand{\SpecialCharTok}[1]{\textcolor[rgb]{0.81,0.36,0.00}{\textbf{#1}}}
\newcommand{\SpecialStringTok}[1]{\textcolor[rgb]{0.31,0.60,0.02}{#1}}
\newcommand{\StringTok}[1]{\textcolor[rgb]{0.31,0.60,0.02}{#1}}
\newcommand{\VariableTok}[1]{\textcolor[rgb]{0.00,0.00,0.00}{#1}}
\newcommand{\VerbatimStringTok}[1]{\textcolor[rgb]{0.31,0.60,0.02}{#1}}
\newcommand{\WarningTok}[1]{\textcolor[rgb]{0.56,0.35,0.01}{\textbf{\textit{#1}}}}
\usepackage{graphicx}
\makeatletter
\def\maxwidth{\ifdim\Gin@nat@width>\linewidth\linewidth\else\Gin@nat@width\fi}
\def\maxheight{\ifdim\Gin@nat@height>\textheight\textheight\else\Gin@nat@height\fi}
\makeatother
% Scale images if necessary, so that they will not overflow the page
% margins by default, and it is still possible to overwrite the defaults
% using explicit options in \includegraphics[width, height, ...]{}
\setkeys{Gin}{width=\maxwidth,height=\maxheight,keepaspectratio}
% Set default figure placement to htbp
\makeatletter
\def\fps@figure{htbp}
\makeatother
\setlength{\emergencystretch}{3em} % prevent overfull lines
\providecommand{\tightlist}{%
  \setlength{\itemsep}{0pt}\setlength{\parskip}{0pt}}
\setcounter{secnumdepth}{-\maxdimen} % remove section numbering
\usepackage{fvextra}
\fvset{breaklines}
\ifLuaTeX
  \usepackage{selnolig}  % disable illegal ligatures
\fi
\usepackage{bookmark}
\IfFileExists{xurl.sty}{\usepackage{xurl}}{} % add URL line breaks if available
\urlstyle{same}
\hypersetup{
  pdftitle={multivariate\_hw6},
  pdfauthor={Na SeungChan},
  hidelinks,
  pdfcreator={LaTeX via pandoc}}

\title{multivariate\_hw6}
\author{Na SeungChan}
\date{2024-12-03}

\begin{document}
\maketitle

\begin{center}\rule{0.5\linewidth}{0.5pt}\end{center}

\section{Q5.1}\label{q5.1}

\begin{Shaded}
\begin{Highlighting}[]
\FunctionTok{pf}\NormalTok{(}\DecValTok{50}\SpecialCharTok{/}\DecValTok{11}\NormalTok{, }\DecValTok{2}\NormalTok{, }\DecValTok{2}\NormalTok{, }\AttributeTok{lower.tail =} \ConstantTok{FALSE}\NormalTok{) }\CommentTok{\# F(2, 2) 계산}
\end{Highlighting}
\end{Shaded}

\begin{verbatim}
## [1] 0.1803279
\end{verbatim}

\begin{Shaded}
\begin{Highlighting}[]
\FunctionTok{qf}\NormalTok{(}\FloatTok{0.05}\NormalTok{, }\DecValTok{2}\NormalTok{, }\DecValTok{2}\NormalTok{, }\AttributeTok{lower.tail =} \ConstantTok{FALSE}\NormalTok{)}
\end{Highlighting}
\end{Shaded}

\begin{verbatim}
## [1] 19
\end{verbatim}

\begin{center}\rule{0.5\linewidth}{0.5pt}\end{center}

\#Q5.3

\begin{Shaded}
\begin{Highlighting}[]
\NormalTok{invS }\OtherTok{\textless{}{-}} \FunctionTok{matrix}\NormalTok{(}\FunctionTok{c}\NormalTok{(}\FloatTok{203.018}\NormalTok{, }\SpecialCharTok{{-}}\FloatTok{163.391}\NormalTok{, }\SpecialCharTok{{-}}\FloatTok{163.091}\NormalTok{, }\FloatTok{200.228}\NormalTok{), }\AttributeTok{nrow =} \DecValTok{2}\NormalTok{)}
\NormalTok{muX }\OtherTok{\textless{}{-}} \FunctionTok{matrix}\NormalTok{(}\FunctionTok{c}\NormalTok{(}\FloatTok{0.014}\NormalTok{, }\FloatTok{0.003}\NormalTok{), }\AttributeTok{nrow =} \DecValTok{2}\NormalTok{)}
\NormalTok{Tsquared\_Q3 }\OtherTok{\textless{}{-}} \DecValTok{42} \SpecialCharTok{*} \FunctionTok{t}\NormalTok{(muX) }\SpecialCharTok{\%*\%}\NormalTok{ invS }\SpecialCharTok{\%*\%}\NormalTok{ muX}
\NormalTok{Tsquared\_Q3}
\end{Highlighting}
\end{Shaded}

\begin{verbatim}
##          [,1]
## [1,] 1.171016
\end{verbatim}

\begin{Shaded}
\begin{Highlighting}[]
\FunctionTok{pf}\NormalTok{(Tsquared\_Q3}\SpecialCharTok{*}\DecValTok{20}\SpecialCharTok{/}\DecValTok{41}\NormalTok{, }\DecValTok{2}\NormalTok{, }\DecValTok{40}\NormalTok{, }\AttributeTok{lower.tail =} \ConstantTok{FALSE}\NormalTok{) }\SpecialCharTok{\textless{}} \FloatTok{0.05} \CommentTok{\# F(2, 40) 계산. not reject H0.}
\end{Highlighting}
\end{Shaded}

\begin{verbatim}
##       [,1]
## [1,] FALSE
\end{verbatim}

\begin{Shaded}
\begin{Highlighting}[]
\NormalTok{critical\_value }\OtherTok{\textless{}{-}} \FunctionTok{qf}\NormalTok{(}\FloatTok{0.05}\NormalTok{, }\DecValTok{2}\NormalTok{, }\DecValTok{40}\NormalTok{, }\AttributeTok{lower.tail =} \ConstantTok{FALSE}\NormalTok{)}\SpecialCharTok{*}\NormalTok{(}\DecValTok{41}\SpecialCharTok{/}\DecValTok{20}\NormalTok{) }\CommentTok{\#기각역의 critical value.}
\end{Highlighting}
\end{Shaded}

\begin{center}\rule{0.5\linewidth}{0.5pt}\end{center}

\#Q5.4

우선 문제풀이에 쓰일 자료를 입력한다.

\begin{Shaded}
\begin{Highlighting}[]
\NormalTok{mu }\OtherTok{\textless{}{-}} \FunctionTok{c}\NormalTok{(}\FloatTok{95.52}\NormalTok{, }\FloatTok{164.38}\NormalTok{, }\FloatTok{55.69}\NormalTok{, }\FloatTok{93.39}\NormalTok{, }\FloatTok{17.98}\NormalTok{, }\FloatTok{31.13}\NormalTok{)}
\NormalTok{va }\OtherTok{\textless{}{-}} \FunctionTok{c}\NormalTok{(}\FloatTok{3266.46}\NormalTok{, }\FloatTok{721.91}\NormalTok{, }\FloatTok{179.28}\NormalTok{, }\FloatTok{474.98}\NormalTok{, }\FloatTok{9.95}\NormalTok{, }\FloatTok{21.26}\NormalTok{)}
\NormalTok{s14 }\OtherTok{\textless{}{-}} \FloatTok{1175.50}
\NormalTok{s56 }\OtherTok{\textless{}{-}} \FloatTok{13.88}
\end{Highlighting}
\end{Shaded}

\subsection{(a)}\label{a}

\begin{Shaded}
\begin{Highlighting}[]
\NormalTok{crit\_a }\OtherTok{\textless{}{-}} \FunctionTok{qchisq}\NormalTok{(}\FloatTok{0.05}\NormalTok{, }\DecValTok{6}\NormalTok{, }\AttributeTok{lower.tail =} \ConstantTok{FALSE}\NormalTok{)}

\ControlFlowTok{for}\NormalTok{ (i }\ControlFlowTok{in} \DecValTok{1}\SpecialCharTok{:}\DecValTok{6}\NormalTok{) \{}
  \FunctionTok{print}\NormalTok{(}\FunctionTok{c}\NormalTok{(mu[i] }\SpecialCharTok{{-}} \FunctionTok{sqrt}\NormalTok{(va[i]}\SpecialCharTok{*}\NormalTok{crit\_a}\SpecialCharTok{/}\DecValTok{61}\NormalTok{), mu[i] }\SpecialCharTok{+} \FunctionTok{sqrt}\NormalTok{(va[i]}\SpecialCharTok{*}\NormalTok{crit\_a}\SpecialCharTok{/}\DecValTok{61}\NormalTok{)))}
\NormalTok{\}}
\end{Highlighting}
\end{Shaded}

\begin{verbatim}
## [1]  69.55347 121.48653
## [1] 152.1728 176.5872
## [1] 49.60667 61.77333
## [1]  83.48823 103.29177
## [1] 16.54687 19.41313
## [1] 29.03513 33.22487
\end{verbatim}

\subsection{(b)}\label{b}

\begin{Shaded}
\begin{Highlighting}[]
\NormalTok{matrix\_s14 }\OtherTok{\textless{}{-}} \FunctionTok{matrix}\NormalTok{(}\FunctionTok{c}\NormalTok{(va[}\DecValTok{1}\NormalTok{], s14, s14, va[}\DecValTok{4}\NormalTok{]), }\AttributeTok{nrow =} \DecValTok{2}\NormalTok{)}
\NormalTok{eigen\_result }\OtherTok{\textless{}{-}} \FunctionTok{eigen}\NormalTok{(matrix\_s14, }\AttributeTok{symmetric =} \ConstantTok{TRUE}\NormalTok{)}

\FunctionTok{c}\NormalTok{(mu[}\DecValTok{1}\NormalTok{] }\SpecialCharTok{{-}} \FunctionTok{sqrt}\NormalTok{(crit\_a}\SpecialCharTok{*}\NormalTok{eigen\_result}\SpecialCharTok{$}\NormalTok{values[}\DecValTok{1}\NormalTok{]}\SpecialCharTok{/}\DecValTok{61}\NormalTok{), }
\NormalTok{  mu[}\DecValTok{1}\NormalTok{] }\SpecialCharTok{+} \FunctionTok{sqrt}\NormalTok{(crit\_a}\SpecialCharTok{*}\NormalTok{eigen\_result}\SpecialCharTok{$}\NormalTok{values[}\DecValTok{1}\NormalTok{]}\SpecialCharTok{/}\DecValTok{61}\NormalTok{))}
\end{Highlighting}
\end{Shaded}

\begin{verbatim}
## [1]  67.90068 123.13932
\end{verbatim}

\begin{Shaded}
\begin{Highlighting}[]
\FunctionTok{c}\NormalTok{(mu[}\DecValTok{4}\NormalTok{] }\SpecialCharTok{{-}} \FunctionTok{sqrt}\NormalTok{(crit\_a}\SpecialCharTok{*}\NormalTok{eigen\_result}\SpecialCharTok{$}\NormalTok{values[}\DecValTok{2}\NormalTok{]}\SpecialCharTok{/}\DecValTok{61}\NormalTok{), }
\NormalTok{  mu[}\DecValTok{4}\NormalTok{] }\SpecialCharTok{+} \FunctionTok{sqrt}\NormalTok{(crit\_a}\SpecialCharTok{*}\NormalTok{eigen\_result}\SpecialCharTok{$}\NormalTok{values[}\DecValTok{2}\NormalTok{]}\SpecialCharTok{/}\DecValTok{61}\NormalTok{))}
\end{Highlighting}
\end{Shaded}

\begin{verbatim}
## [1] 90.31119 96.46881
\end{verbatim}

\subsection{(c)}\label{c}

\begin{Shaded}
\begin{Highlighting}[]
\NormalTok{crit\_b }\OtherTok{\textless{}{-}} \FunctionTok{qt}\NormalTok{(}\DecValTok{1}\SpecialCharTok{/}\DecValTok{240}\NormalTok{, }\DecValTok{60}\NormalTok{, }\AttributeTok{lower.tail =} \ConstantTok{FALSE}\NormalTok{)}

\ControlFlowTok{for}\NormalTok{ (i }\ControlFlowTok{in} \DecValTok{1}\SpecialCharTok{:}\DecValTok{6}\NormalTok{) \{}
  \FunctionTok{print}\NormalTok{(}\FunctionTok{c}\NormalTok{(mu[i] }\SpecialCharTok{{-}}\NormalTok{ crit\_b}\SpecialCharTok{*}\FunctionTok{sqrt}\NormalTok{(va[i]}\SpecialCharTok{/}\DecValTok{61}\NormalTok{), mu[i] }\SpecialCharTok{+}\NormalTok{ crit\_b}\SpecialCharTok{*}\FunctionTok{sqrt}\NormalTok{(va[i]}\SpecialCharTok{/}\DecValTok{61}\NormalTok{)))}
\NormalTok{\}}
\end{Highlighting}
\end{Shaded}

\begin{verbatim}
## [1]  75.55331 115.48669
## [1] 154.9934 173.7666
## [1] 51.01229 60.36771
## [1]  85.77614 101.00386
## [1] 16.87801 19.08199
## [1] 29.51917 32.74083
\end{verbatim}

\subsection{(d)}\label{d}

\begin{Shaded}
\begin{Highlighting}[]
\FunctionTok{c}\NormalTok{(mu[}\DecValTok{1}\NormalTok{] }\SpecialCharTok{{-}}\NormalTok{ crit\_b}\SpecialCharTok{*}\FunctionTok{sqrt}\NormalTok{(va[}\DecValTok{1}\NormalTok{]}\SpecialCharTok{/}\DecValTok{61}\NormalTok{), mu[}\DecValTok{1}\NormalTok{] }\SpecialCharTok{+}\NormalTok{ crit\_b}\SpecialCharTok{*}\FunctionTok{sqrt}\NormalTok{(va[}\DecValTok{1}\NormalTok{]}\SpecialCharTok{/}\DecValTok{61}\NormalTok{))}
\end{Highlighting}
\end{Shaded}

\begin{verbatim}
## [1]  75.55331 115.48669
\end{verbatim}

\begin{Shaded}
\begin{Highlighting}[]
\FunctionTok{c}\NormalTok{(mu[}\DecValTok{4}\NormalTok{] }\SpecialCharTok{{-}}\NormalTok{ crit\_b}\SpecialCharTok{*}\FunctionTok{sqrt}\NormalTok{(va[}\DecValTok{4}\NormalTok{]}\SpecialCharTok{/}\DecValTok{61}\NormalTok{), mu[}\DecValTok{4}\NormalTok{] }\SpecialCharTok{+}\NormalTok{ crit\_b}\SpecialCharTok{*}\FunctionTok{sqrt}\NormalTok{(va[}\DecValTok{4}\NormalTok{]}\SpecialCharTok{/}\DecValTok{61}\NormalTok{))}
\end{Highlighting}
\end{Shaded}

\begin{verbatim}
## [1]  85.77614 101.00386
\end{verbatim}

\subsection{(e)}\label{e}

\begin{Shaded}
\begin{Highlighting}[]
\NormalTok{crit\_c }\OtherTok{\textless{}{-}} \FunctionTok{qt}\NormalTok{(}\DecValTok{1}\SpecialCharTok{/}\DecValTok{280}\NormalTok{, }\DecValTok{60}\NormalTok{, }\AttributeTok{lower.tail =} \ConstantTok{FALSE}\NormalTok{)}

\FunctionTok{c}\NormalTok{(mu[}\DecValTok{6}\NormalTok{] }\SpecialCharTok{{-}}\NormalTok{ mu[}\DecValTok{5}\NormalTok{] }\SpecialCharTok{{-}}\NormalTok{ crit\_c}\SpecialCharTok{*}\FunctionTok{sqrt}\NormalTok{((va[}\DecValTok{5}\NormalTok{]}\SpecialCharTok{{-}}\DecValTok{2}\SpecialCharTok{*}\NormalTok{s56}\SpecialCharTok{+}\NormalTok{va[}\DecValTok{6}\NormalTok{])}\SpecialCharTok{/}\DecValTok{61}\NormalTok{), }
\NormalTok{  mu[}\DecValTok{6}\NormalTok{] }\SpecialCharTok{{-}}\NormalTok{ mu[}\DecValTok{5}\NormalTok{] }\SpecialCharTok{+}\NormalTok{ crit\_c}\SpecialCharTok{*}\FunctionTok{sqrt}\NormalTok{((va[}\DecValTok{5}\NormalTok{]}\SpecialCharTok{{-}}\DecValTok{2}\SpecialCharTok{*}\NormalTok{s56}\SpecialCharTok{+}\NormalTok{va[}\DecValTok{6}\NormalTok{])}\SpecialCharTok{/}\DecValTok{61}\NormalTok{))}
\end{Highlighting}
\end{Shaded}

\begin{verbatim}
## [1] 12.48757 13.81243
\end{verbatim}

\begin{center}\rule{0.5\linewidth}{0.5pt}\end{center}

\section{Q5.5}\label{q5.5}

우선, 문제에 제시된 정보를 R 코드로 입력한다. q+1 = k = 3개 범주로부터
다항분포에 따라 다음과 같이 분포가 생성되었다.

\[
X \; {\sim} \; Multi(n = 200, ~ p = (p_1, p_2, p_3))
\]

이제 Result의 결과를 이에 적용한다.

\begin{Shaded}
\begin{Highlighting}[]
\NormalTok{p1h }\OtherTok{\textless{}{-}} \DecValTok{117}\SpecialCharTok{/}\DecValTok{200}
\NormalTok{p2h }\OtherTok{\textless{}{-}} \DecValTok{62}\SpecialCharTok{/}\DecValTok{200}
\NormalTok{p3h }\OtherTok{\textless{}{-}} \DecValTok{21}\SpecialCharTok{/}\DecValTok{200}

\NormalTok{ph }\OtherTok{\textless{}{-}} \FunctionTok{c}\NormalTok{(p1h, p2h, p3h)}
\NormalTok{sh }\OtherTok{\textless{}{-}} \FunctionTok{matrix}\NormalTok{(}\FunctionTok{c}\NormalTok{(p1h}\SpecialCharTok{\^{}}\DecValTok{2}\NormalTok{, }\SpecialCharTok{{-}}\NormalTok{p1h}\SpecialCharTok{*}\NormalTok{p2h, }\SpecialCharTok{{-}}\NormalTok{p1h}\SpecialCharTok{*}\NormalTok{p3h, }\SpecialCharTok{{-}}\NormalTok{p1h}\SpecialCharTok{*}\NormalTok{p2h, p2h}\SpecialCharTok{\^{}}\DecValTok{2}\NormalTok{, }\SpecialCharTok{{-}}\NormalTok{p1h}\SpecialCharTok{*}\NormalTok{p3h, }\SpecialCharTok{{-}}\NormalTok{p1h}\SpecialCharTok{*}\NormalTok{p3h, p2h}\SpecialCharTok{*}\NormalTok{p3h, p3h}\SpecialCharTok{\^{}}\DecValTok{2}\NormalTok{), }\AttributeTok{nrow =} \DecValTok{3}\NormalTok{)}

\NormalTok{crit }\OtherTok{\textless{}{-}} \FunctionTok{qchisq}\NormalTok{(}\FloatTok{0.05}\NormalTok{, }\DecValTok{2}\NormalTok{, }\AttributeTok{lower.tail =} \ConstantTok{FALSE}\NormalTok{)}
\end{Highlighting}
\end{Shaded}

따라서 \(p_1, p_2, p_3\) 각각의 simultaneous 95\% CI는 다음과 같이
주어진다.

\begin{Shaded}
\begin{Highlighting}[]
\FunctionTok{c}\NormalTok{(p1h }\SpecialCharTok{{-}} \FunctionTok{sqrt}\NormalTok{(crit}\SpecialCharTok{*}\NormalTok{sh[}\DecValTok{1}\NormalTok{]}\SpecialCharTok{/}\DecValTok{200}\NormalTok{), p1h }\SpecialCharTok{+} \FunctionTok{sqrt}\NormalTok{(crit}\SpecialCharTok{*}\NormalTok{sh[}\DecValTok{1}\NormalTok{]}\SpecialCharTok{/}\DecValTok{200}\NormalTok{))}
\end{Highlighting}
\end{Shaded}

\begin{verbatim}
## [1] 0.4837471 0.6862529
\end{verbatim}

\begin{Shaded}
\begin{Highlighting}[]
\FunctionTok{c}\NormalTok{(p2h }\SpecialCharTok{{-}} \FunctionTok{sqrt}\NormalTok{(crit}\SpecialCharTok{*}\NormalTok{sh[}\DecValTok{5}\NormalTok{]}\SpecialCharTok{/}\DecValTok{200}\NormalTok{), p2h }\SpecialCharTok{+} \FunctionTok{sqrt}\NormalTok{(crit}\SpecialCharTok{*}\NormalTok{sh[}\DecValTok{5}\NormalTok{]}\SpecialCharTok{/}\DecValTok{200}\NormalTok{))}
\end{Highlighting}
\end{Shaded}

\begin{verbatim}
## [1] 0.2563446 0.3636554
\end{verbatim}

\begin{Shaded}
\begin{Highlighting}[]
\FunctionTok{c}\NormalTok{(p3h }\SpecialCharTok{{-}} \FunctionTok{sqrt}\NormalTok{(crit}\SpecialCharTok{*}\NormalTok{sh[}\DecValTok{9}\NormalTok{]}\SpecialCharTok{/}\DecValTok{200}\NormalTok{), p3h }\SpecialCharTok{+} \FunctionTok{sqrt}\NormalTok{(crit}\SpecialCharTok{*}\NormalTok{sh[}\DecValTok{9}\NormalTok{]}\SpecialCharTok{/}\DecValTok{200}\NormalTok{))}
\end{Highlighting}
\end{Shaded}

\begin{verbatim}
## [1] 0.08682641 0.12317359
\end{verbatim}

\end{document}
